\documentclass{article}
\usepackage[utf8]{inputenc}
\usepackage[vietnam]{babel}
 
\usepackage[
backend=biber,
style=numeric,
sorting=ynt
]{biblatex}
 
\addbibresource{references.bib}
\begin{document}
\begin{center}
\textbf{PHẦN TÓM TẮT}\\
\vspace{10mm}
\textbf{TÊN ĐỀ TÀI (tiếng Việt)}
\end{center}

\vspace{10mm}
Mục tiêu của đề tài là xây dựng một hệ thống có thể nhận diện kí tự số viết tay.

\vspace{6mm}

So sánh cuộc đấu cờ giữa Garry Kasparov và Deep Blue \cite{deepblue} năm 1996 và AlphaGo năm 2016 \cite{alphago}. Trong vòng 20 năm có tới 2 chương trình dùng để so tài giữa máy móc và con người. Nhưng thực tế Deep Blue được IBM lập trình cho nhiều nước cờ, trong quá trình thi đấu thì nó sẽ chọn ra nước cờ thích hợp; trong khi đó AlphaGo là một trí tuệ nhân tạo (A.I.) được tạo ra và nó cải thiện tính năng nó bằng cách tự tạo ra tình thế rồi giải quyết tình huống đó rồi sẽ học cách đi nước cờ. AlphaGo hơn Deep Blue ở chỗ là nếu Deep Blue cho chơi loại cờ khác như cờ tướng thì nó sẽ không thể nào chơi được, còn AlphaGo thì nó sẽ tự học từ đầu và sẽ chơi được bất cứ thể loại cờ nào.

\vspace{6mm}

Đề tài \emph{Nhận diện kí tự dùng trí tuệ nhân tạo} so với phương pháp nhận diện thông thường dùng thị giác máy tính cũng giống như việc so sánh giữa AlphaGo và Deep Blue. Lấy ví dụ giờ ta không muốn nhận diện kí tự nữa mà chúng ta lại muốn nhận diện vật khác ví dụ như phân biệt chó hay mèo thì chúng ta chỉ cần chỉnh lại cấu trúc của hệ thống, thay đổi dữ liệu và cho nó tự học lại là giải quyết được vấn đề; trong khi so với phương pháp thị giác máy tính thông thường thì ta phải tìm lại đặc điểm của con chó ra sao, đặc điểm của con mèo ra sao rồi phân biệt chúng dựa trên đặc điểm đó.

\vspace{6mm}

Với lí do trên đề tài này sẽ trình bày phương pháp chung cho việc nhận dạng, đi xây dựng thuật toán và phân tích các thuật toán. Cho nên sẽ lấy ví dụ là nhận dạng kí tự số viết tay.
 
\medskip
 
\printbibliography
\end{document}